\documentclass{coupled2019_abstract}

%\usepackage{graphicx}
%\usepackage{amsmath}
%\usepackage{amsfonts}
%\usepackage{amssymb}

\usepackage{hyperref}

\hypersetup{
    bookmarks=true,
    breaklinks=true,
    bookmarksopen=true,
    pdftitle={Robust polyhedral ... real-world geometries ... the AggFEM method},    % title
    pdfauthor={Francesc Verdugo},     % author
    colorlinks=true,       % false: boxed links; true: colored links
    linkcolor=black,          % color of internal links (change box color with linkbordercolor)
    citecolor=black,        % color of links to bibliography
    filecolor=black,      % color of file links
    urlcolor=black           % color of external links
}

\title{ Robust polyhedral ... real-world geometries ... the AggFEM method }

\author{Pere Antoni Martorell$^{*\dag}$, Santiago Badia$^{\dag\ddag}$ and Francesc Verdugo$^{\dag}$ }

\address{
$^{*}$ Department of Civil and Environmental Engineering \\
Universitat Polit\`{e}cnica de Catalunya\\
Building C1, Campus Nord UPC\\
Jordi Girona, 1-3, 08034 Barcelona, Spain
\and
$^{\dag}$ Centre Internacional de Metodes Numerics en Enginyeria (CIMNE)\\
Building C3, Parc Mediterrani de la Tecnologia \\
Esteve Terradas, 5, 08860 Castelldefels, Spain\\
e-mails: \{ pmartorell, fverdudo, sbadia \}@cimne.upc.edu
\and
$^{\ddag}$
School of Mathematics\\
Monash University\\
Clayton, Victoria, 3800, Australia
}


\begin{document}
%\maketitle
\begin{center}
\bf ABSTRACT
\end{center}

Unfitted finite element methods are useful techniques to simulate problems defined on 3D complex domains. 
In this context, the conventional approach is to represent the problem geometry using level-set methods. 
Geometrical data based on level-set functions allow efficient procedures (usually based on marching cubes algorithms) for the generation of integration cells in cut elements. 
However, real-world engineering applications consider often 3D CAD data for the geometrical definitions. 
This makes challenging the usage of standard unfitted techniques, since there is not a general and accurate way to translate 3D CAD models into level-set functions.

In this work, we explore a novel technique in order to generate integration grids in cut cells.
In contrast to level-set methods, our methodology can be robustly feed from first order CAD models, e.g., STLs.
The used approach is based on robust polyhedral clipping \cite{Powel}, capturing the exacty even non-convex geometries.
This method is extensible to high order geometries, higher dimensions and parallelizable at large scale.

The techinque is implementated in the framework of finite element package Gridap \cite{Gridap} and it have been tested over a large subset of \cite{10k} with the AggFEM method \cite{AgFEM}.

% cite julia lang (implicit on Gridap)?
% cite high oder? high dims?

\begin{thebibliography}{99}

\bibitem{Powel}
D. Powell and T. Abel. 
An exact general remeshing scheme applied to physically conservative voxelization.
\textit{Journal of Computational Physics}.
(2015) \textbf{297}: 340-356. 
\url{doi.org/10.1016/j.jcp.2015.05.022}

\bibitem{Gridap} 
S. Badia, F. Verdugo.
Gridap: An extensible Finite Element toolbox in Julia.
\textit{Journal of Open Source Software}. 
(2020) \textbf{5}(52): 2520.
\url{doi.org/10.21105/joss.02520}

\bibitem{10k}
Q. Zhou and A. Jacobson. 
Thingi10K: A Dataset of 10,000 3D-Printing Models.
(2016).
\url{arxiv.org/abs/1605.04797}

\bibitem{AgFEM} 
S. Badia, F. Verdugo, A. F. Mart\'in.
The aggregated unfitted finite element method for elliptic problems.
\textit{Computer Methods in Applied Mechanics and Engineering}. 
(2018) \textbf{336}: 533--553. 
\url{doi.org/10.1016/j.cma.2018.03.022}


\end{thebibliography}

\end{document}


